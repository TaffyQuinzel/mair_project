\subsection{Reflection on work}
The general division of task among team members was as follows: Sophie and Marjolein worked mainly on the literature review, and Maayan and Louis worked mainly on the experiment set-up and data analysis. However, we also helped each other on these tasks, and we had a lot of discussions on both aspects of the project. Therefore, the division is not that clear-cut. During the project weeks, we had a meeting with the whole group twice a week. In such meetings, we discussed the process so far and results individual team members had found. Moreover, we prepared and executed the experiment all together. We have spent a full day together on acquiring participants and performing the experiment.

Some of the group members did not have any background in experimental research. This project has provided them with more experience on experiment design and data analysis. Other group members did not have much background in philosophy in general, and thus also no background on philosophy of the mind and AI. This project has provided them with more knowledge on the current debate on AI, and with experience on how to write a philosophical literature review. As a group, we learned how to work together with people from different backgrounds which have different views on how to conduct research. Therefore, we also learned how to conduct research in different domains. We thus conclude that performing this project at the start of the AI program is valuable.

As this reflection is used to assess whether everyone pulled their weight, we would like to note that we think this is indeed true, i.e. everyone in our group has pulled their weight during the process.
