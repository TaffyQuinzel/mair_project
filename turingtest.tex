\section{Turing Test}
In 1950, Alan Turing elaborated his ideas about the Turing Test proposed an early version of the Turing Test in a paper titled ‘Computing Machinery and Intelligence’~\cite{turing1950computing}. The initial question proposed by Turing was whether machines can think. Turing argued that rather than discussing the definition of the words ‘machine’ and ‘think’, the question can be answered by executing an experiment. The original experiment, called the ‘Imitation Game’, consists of a man, a machine, and an interrogator. The interrogator is located in a different room than the man and the machine, and his goal is to identify which of the two others is a man and which is a woman. The interrogator can do so by asking questions via some chat application to both the man and machine simultaneously. The goal of the man is trying to cause the interrogator to make a wrong identification, thus that the interrogator would mistake him for being a woman. The machine, which is clearly neither male nor female, has as goal to resemble a woman in order to cause the interrogator to classify it as a woman. If the interrogator decides wrongly as often as when the game is played with a woman in place of the machine, Turing would say that the machine is intelligent.

Over the years, this original experiment has been slightly changed into what we nowadays call the ‘Turing Test’. In the contemporary version of the Turing Test, the interrogator does not have to decide which of the two is the man and which is the woman, but rather which of the two is the human and which is the machine. If we accept the validity of the test, we would call a machine intelligent when the interrogator decides wrongly as often as correct when the experiment is executed multiple times.
