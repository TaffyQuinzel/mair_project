\section{Turing Test}
In 1950, Alan Turing elaborated his ideas about the Turing Test in his paper titled \textit{Computing Machinery and Intelligence}~\cite{turing1950computing}. The initial question proposed by Turing was whether machines can think. Turing states that this question is in itself \textit{too meaningless} to deserve discussion, since we don't know the exact meanings of the words \textit{machine} and \textit{think}. Turing argued that rather than discussing the definition of these words, a more precise question (which is certainly related to the original question) can be answered by executing an experiment. The experiment is based on a game, called the \textit{Imitation Game}. This game is played with a man, a woman, and an interrogator (which can be either man or woman). The interrogator is located in a different room than the man and the woman, and his goal is to identify which one is a man and which one is a woman. The interrogator can do so by asking questions by teletype to both the man and the woman simultaneously. The goal of the man is trying to cause the interrogator to make a wrong identification, thus that the interrogator would mistake him for being a woman.

Turing suggests that when we replace the man in the game with a computer we can measure how intelligent the machine is. The machine, which is clearly neither male nor female, has the goal to resemble a woman in order to cause the interrogator to classify it as a woman. If the interrogator decides wrongly as often as when the game is played with a man in place of the machine, Turing would say that the machine is intelligent.

Over the years, this original experiment has been slightly changed into what we nowadays call the \textit{Turing Test}. In the contemporary version of the Turing Test, the interrogator does not have to decide which of the two is the man and which is the woman, but rather which of the two is the human and which is the machine. If we accept the validity of the test, we would call a machine intelligent when the interrogator decides wrongly as often as correct when the experiment is executed multiple times.

It is interesting to note that Turing himself predicted in his article that around the year 2000 (fifty years after his original paper) machines with a storage capacity of about 109 (the ‘storage capacity’ of a machine, as defined by Turing, is the logarithm to the base two of the number of states of the machine) would play the game so well that an average interrogator will not have more than a 70\% chance of making the right identification after five minutes of questioning. Although, some have claimed that their chatbots have indeed achieve this at~\cite{copeland2014eugene}, most research into the Turing test and the outcome of the annual competition of the Loebner's prize show that we are far from getting close to passing the Turing Test.
