\section{Objections to the original Turing Test}

\cite{sep-turing-test,another}
\paragraph{The first objection} is that the Turing Test is too hard for a computer to pass. Right now, many people think that we are so far away from passing the test that the goal of passing it might not be a realistic goal in A.I. research. Some people even think that no machine that man creates will ever pass the test.

One of the reasons that the test is too hard was given by Robert M. French who thinks that nothing without a “human subcognitive substrate” could ever pass the test. According to French, there are obvious questions that people can use to discriminate between humans and computers, which reveal in his words “low-level cognitive structure”. By low-level cognitive structure he means the subconscious associative network in human minds. Humans develop many associations during their lives: they learn through experience that certain words or concepts are more commonly used together with second words or concepts than others, for example the words ‘bread’ and ‘butter’ in comparison with ‘bread’ and ‘dog’. The interrogator in the Turing Test can make use of this associative network that he shares with other humans while he doesn’t share this (at least not extensively) with the computer.

French expand these ideas and introduced ‘rating games’ which are intentionally designed to be able to distinguish between humans and computers. One of his examples is that an interrogator would ask the participants to rate the name ‘flugbots’ as an appropriate name for a breakfast cereal. The human participant in the game would

This relates to another reason that the test is too hard, which is that it does not only measure intelligence, but also how humanlike the machine is. It might be so that it is particularly hard to simulate certain human features, which have nothing or little to do with intelligence.


\paragraph{The second objection} is not aimed at the difficulty of the test, but at the scope of the test, as it states that the Turing Test is too narrow for testing intelligence. The most known argument for this objection is given by John Searle (BRON3), which we will discuss first. After Searle’s famous objection, we will discuss arguments supporting this objection made by other authors.

Searle’s main argument is described in his paper called ‘Mind, brains, and programs’ (BRON3) and describes the Chinese Room argument. The argument goes as follows. Suppose we have a computer program called ‘Sam’ which is able to answer Chinese questions back in Chinese. We now replace this program ‘Sam’ by a human ‘Joe’, and we envision this situation as if Joe is sitting in a room detached from the rest of the world. Joe is equipped with instruction books on the program instructions, thus the books describe which character to add in Joe’s answer when a specific character is present in the question. Joe is American and he does not understand anything from Chinese. Therefore, Joe has no idea what the questions or answers mean. However, to an external observer, it looks like that the program is intelligent as it gives reasonable answers to the questions.

The claim Searle wants to make with this thought experiment is that executing a program does not imply any understanding of what the program is doing or attaching meaning to the Chinese symbols which are used. A computer program uses symbols (zeros and ones) in a similar way, thus is also not capable of understanding these symbols or of attaching meaning to the symbols. Because the Turing Test only uses words, which can be interpreted as symbols, the test does not test real understanding.

Searle’s objection is related to the ‘Sense organs objection’ given by Copeland (BRON2). This objections concerns the fact that the test is only aimed at question answering, and it does not test whether the computer can relate the used words to things in the world. According to this objection, a computer could thus pass the test without having any understanding of the words he is using. An alteration for the Turing Test is given, and argues that the test should be strengthened by providing the computer artificial sense organs such as vision and speech, which can be used to test the computer’s understanding of the meaning of the words.

The question then remains whether giving a computer artificial sense organs could solve Searle’s objection. We can argue that the computer now not only works with letters and words, which are coded as zeros and ones in the computer, so more thorough understanding is now possible. On the other hand, we can argue that images obtained by vision or sounds that the computer hears or produces can also be encoded into zeros and ones. Thus, images and sounds are in the end also symbols and adding those sense organs to the Turing Test would in no way resolve Searle’s objection.

From a less analytical and more practical point of view, Gerald J. Erion (BRON4) also argues that while computers might be able to pass the test, they can still not do much else than the limited tasks involved in passing the test. During the test, computers are only answering questions via some chat interface, which is a very limited task. We question whether his statement is correct. When passing the test, the computer must be able to solve a wide variety of every circumstances, related to common knowledge, memory, personal identity, and many more. Although the Turing Test is text-only, it thus requires the computer to do many subtasks which contribute to the computer’s credibility of being a human.
