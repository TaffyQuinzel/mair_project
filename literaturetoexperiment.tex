\section{From Literature to Experiment}

One of such functions of the software, namely automatically adding spelling errors to the computer’s answer, will be evaluated in our experiment. By adding this manipulation, we are going to investigate whether we can modify the Turing Test in favour of the computer. In this way, we will also try to mitigate the first objection which states that the test is too hard for a computer to pass. Moreover, in line with the first alteration, we will also measure the participant’s belief which of the two conversation partners is a computer on a probability scale. Therefore, we measure the outcome on a ratio scale instead of on a simple nominal scale.

When designing or executing a Turing Test, the output criterion~\cite{copeland2015artificial} is important to take into account. The output criterion states that the interrogator should talk to the human and machine simultaneously, in order to be able to compare the conversational output of them both when assessing which is the human and which is the machine. Therefore, we make sure to satisfy this criterion in our execution of the test.
