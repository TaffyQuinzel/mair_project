\section{Introduction}
We live in a world in which every year technology gets smarter and robots capable of more human tasks. Therefore, as Artificial Intelligence researchers, we often get asked (by both academic and non-academics) what this will entail for the future. Will the differences between humans and machines diminish? Will a machine (or a robot if you like) ever be able to truly think like a human being? The first real attempt of a practical way to test whether machines can think, came from computer scientist Alan Turing in 1936. He proposed a test in which one person (the so-called interrogator) talks simultaneously with two others, out of which one is a computer and the other one is a human being. When the interrogator makes the wrong identification (in which he identifies a computer as a human) at least half of the times, the computer passes the test and the machine is said to think.

80 years after Turing's original ideas, we would like to evaluate the following research question: how can and should a Turing Test be used in future Artificial Intelligence research. Obviously, there have been many objections against the whole idea of the test. We will firstly focus on two theoretical objections against the test. Afterwards we focus on two more practical objections that question whether the Turing Test could be practically used in science. In the light of these objections, we will discuss several alterations on the tests proposed by AI researchers. Finally, we will test some alterations to the Turing Test in practice by executing the Turing Test with a small manipulation ourselves. This experiment will provide some insights into the performance of the Turing Test. Moreover, the validity of the Turing Test will be discussed based on the outcomes of the experiment.

The research question of this paper is at the core of AI and therefore cannot be addressed by disciplinary viewpoints from either psychology, philosophy or computer science alone. Instead, it should be addressed with an interdisciplinary approach. Critical analyzing that is being done in philosophy helps understanding the central concepts and problems. Setting up an experiment and analysing the data using statistical methods needs to be learned from psychology. Finally, manipulating an experiment uses skills and tools from computer science. Therefore, we conclude that in answering our research question we need an integrative perspective that combines (at least) those three mentioned disciplines. To evaluate the validity of the Turing Test is both of theoretical and practical importance. It is practically important because it helps us in setting appropriate goals for AI research. It is theoretically important since a negative outcome (that passing a Turing Test should not be an appropriate goal in AI) has the theoretical implication that it might not be relevant to keep discussing the philosophical problems related to the test.
