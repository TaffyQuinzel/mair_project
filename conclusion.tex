\section{Conclusion}
The main research question of this paper is: How can and should a Turing Test be used in future Artificial Intelligence research? We have seen that there are philosophical objections that argue that the test cannot be used to measure machine intelligence. Although there are no definite conclusions to be drawn, we conclude that those objections are not very relevant in AI research today, since it probably will not be possible for any AI to pass the Turing Test in the foreseeable future.

Therefore, we have argued that the practical objections are more relevant to evaluate whether the Turing Test could be used in AI research. With the experiment, we have shown that the performance of chatbots can improve when we provide them with typical human features, like spelling errors. While our results were not statistically significant, we base this claim on qualitative analysis of the participants responses and the performance of the chatbots. We also altered the original test in such a way that participants were able to identify either one of their conversation partners as the computer, on a probability scale, indicating how certain they were of their choice. The results of the experiment are now more fine-grained. As part of future work, we could investigate more ways to alter the test in order to improve the chatbot’s performance. Even though a better performance does not show that the chatbot becomes more intelligent, it can surely show that it becomes more human-like. Therefore, the Turing Test could be used in AI to understand what is crucial for human language interaction, specifically to understand what is crucial in human chat communication.

Lastly, we have discussed the practical objection that the Turing Test is too narrow to measure intelligence. Even though this was not the main focus of our research, we can still give a few remarks on this objection. As we have seen, a chatbot needs to posses many capabilities in order to perform well in a Turing Test, including having a memory; having common knowledge; having a personal identity etc. It can still be argued that these abilities are not enough to conclude that chatbots are getting intelligent. For now, it is not quite clear that alternative (more demanding) tests are better in measuring machine intelligence. Therefore, we suggest that future research in the Turing Test can help to understand how we can make chatbots (and also robots) more intelligent, even if this is still a narrow interpretation of what it means to be intelligent.
